\documentclass[12pt]{article}

\usepackage[margin=1in]{geometry}
\usepackage{setspace}
\usepackage{hyperref}

\title{Snapshot Objectives\\[0.5em]
SportHub – Multi-Sport Scores \& Favorites Tracker}
\author{Team 4: Miguel, Oscar, Florencio, Jesus}
\date{Fall 2025 – CS3338}

\begin{document}

\maketitle
\tableofcontents
\newpage

\section{Start Objective (Snapshot 1)}

\subsection{Goals at the Start of the Project}
At the start of the project, the main goal for SportHub was to set up the
foundations of the system and the toolchain required by CS3338. More
specifically, our objectives were:

\begin{itemize}
    \item Define the basic idea for SportHub: a multi-sport web app that shows schedules and allows users to track favorite teams.
    \item Set up the GitHub repository and agree on the initial folder structure for frontend, backend, and documentation.
    \item Create the first version of the core documents (SRS, SDD, User Manual, Design Spec, Snapshot Objectives) in \texttt{.tex} format.
    \item Build a simple All Games schedule view using mock or seeded data served from the backend.
    \item Scaffold our tooling: connect Jira, create the first sprint/epics, and prepare TestRail and Docker for later snapshots.
\end{itemize}

\subsection{Reflection}
By the end of Snapshot~1 we had a clear project direction and the basic
architecture in place (React frontend, Node/Express backend, PostgreSQL DB, and
Docker). The schedule view was simple but working with mock data, and the main
documents existed in draft form. This gave us a good starting point to add more
features in later snapshots.

\newpage

\section{1st Checkpoint Objective (Snapshot 2)}

\subsection{Goals for Snapshot 2}
The focus for Snapshot~2 was to move beyond a static prototype and make the
schedule view more useful and realistic. Our main objectives were:

\begin{itemize}
    \item Wire the All Games schedule view to the backend API instead of hardcoded data.
    \item Implement sport and date filters so users can narrow down the list of games.
    \item Stabilize the Docker Compose setup so frontend and backend can be started together.
    \item Create and run initial TestRail test runs for the schedule and filters.
    \item Update SRS, SDD, User Manual, and Design Spec to match the schedule/filter functionality delivered.
\end{itemize}

\subsection{Reflection}
By the end of Snapshot~2, the schedule view felt more like a real application.
Filters were working, and we had a smoother local setup thanks to Docker
Compose. TestRail had its first test run covering basic game retrieval and
filter behavior. Some features (like favorites and admin tools) were still
planned but not implemented yet, which became the focus for Snapshot~3.

\newpage

\section{2nd Checkpoint Objective (Snapshot 3)}

\subsection{Goals for Snapshot 3}
The main objective for Snapshot~3 was to introduce the Favorites flow and begin
personalizing the experience for fans. Our goals were:

\begin{itemize}
    \item Add a Favorites UI on the All Games or Teams page (star/toggle) to let fans select favorite teams.
    \item Implement a \textbf{My Favorites} page where users can see upcoming games for their favorite teams.
    \item Design and partially implement favorites-related API endpoints
          (e.g., \texttt{/api/favorites}, \texttt{/api/favorites/games}).
    \item Extend TestRail test cases to cover adding/removing favorites and viewing My Favorites.
    \item Update SRS, SDD, User Manual, and Design Spec to describe the Favorites feature and new pages.
\end{itemize}

\subsection{Reflection}
Snapshot~3 shifted SportHub from just a schedule viewer into something more
personalized. Even if some favorites logic was still stored on the frontend
temporarily, the UI and flow were in place. We captured these changes in the
documents and TestRail. The remaining work for the final snapshot focused on
polishing, closing gaps, and making the overall system feel complete.

\newpage

\section{Due Date Checkpoint (Snapshot 4 – Final)}

\subsection{Goals for the Final Snapshot}
The last snapshot focused on polishing, documentation, and final verification.
Our main objectives were:

\begin{itemize}
    \item Finalize the Favorites flow and My Favorites page so they behave consistently with the design.
    \item Clean up navigation and filters so the main pages feel unified and easy to use.
    \item Ensure that Docker Compose can start the system with one command and that the basic paths work as expected.
    \item Run final TestRail test runs for the completed features (schedule, filters, favorites) and export reports.
    \item Bring all documents (SRS, SDD, User Manual, Design Spec, Snapshot Objectives) up to date with the final implementation.
\end{itemize}

\subsection{Reflection and Future Work}
By the due date, SportHub met the main goals of the project: a multi-sport
schedule viewer with filters, favorites, and a clear architecture supported by
Jira, TestRail, Docker, and LaTeX documentation. There are still many possible
extensions, such as live scores from external APIs, richer statistics, better
mobile responsiveness, and notifications for upcoming games. However, the
current version provides a solid foundation that demonstrates the intended
software engineering skills for CS3338.

\end{document}
