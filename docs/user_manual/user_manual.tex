\documentclass[12pt]{article}
\usepackage[margin=1in]{geometry}
\usepackage{hyperref}

\begin{document}

\section*{SportHub User Manual}

\section{Introduction}

SportHub is a web application that lets users view upcoming games for
multiple sports and keep a personal list of favorite games. The main users
are sports fans who want one place to quickly see games and track their
favorites.

This manual explains the goals of SportHub, why the software matters, and
how to access and use it during development.

\section{Objectives of SportHub}

\begin{itemize}
  \item Show a list of upcoming games for different sports (NBA, NFL, MLB, etc.).
  \item Allow the user to filter games by sport and date.
  \item Allow the user to mark games as favorites and view a ``My Favorites''
        list (planned for later snapshots).
  \item Provide a simple admin view so project members can add or edit games
        without touching the database directly (planned for later snapshots).
\end{itemize}

\section{Why this software matters}

Many sports fans jump between different apps and websites to check schedules
for different leagues. SportHub aims to collect multiple sports into a single
view and let users focus on the games that matter most to them through the
favorites feature. This makes it faster and easier to keep track of favorite
teams and important matchups.

\section{Accessing SportHub}

\subsection*{Requirements}

\begin{itemize}
  \item Modern web browser (Chrome, Firefox, Edge, etc.).
  \item Local development environment with Node.js and Git (for running the
        project on a laptop).
\end{itemize}

\subsection*{GitHub Repository}

The source code and documentation are stored in the GitHub repository:

\begin{itemize}
  \item GitHub: \url{https://github.com/uchiha1121/cs3338-final-sportapp
}
\end{itemize}

\subsection*{Jira Link}

Project tasks and sprints are tracked in Jira:

\begin{itemize}
  \item Jira board: \url{<PASTE-YOUR-JIRA-LINK-HERE>} ----------------------Still need jira 
\end{itemize}

\subsection*{How to run the project (development)}

During early snapshots the project is still under development, but the
general steps to run it locally are:

\begin{enumerate}
  \item Clone the GitHub repository:
        \begin{verbatim}
        git clone <your-repo-url>
        \end{verbatim}
  \item Set up the backend (Node/Express) in the \texttt{backend} folder.
  \item Set up the frontend in the \texttt{frontend} folder.
  \item In later snapshots, start all services together using
        \texttt{docker-compose up}.
\end{enumerate}

When fully set up, the user will open a browser and navigate to the
frontend URL (for example \texttt{http://localhost:3000}) to use SportHub.

\section{Using SportHub}

\subsection*{Navigation}

All pages share a top navigation bar with links:

\begin{itemize}
  \item \textbf{All Games} -- main page showing upcoming games.
  \item \textbf{My Favorites} -- page showing only favorited games
        (planned for later snapshots).
  \item \textbf{Admin} -- page for managing games (planned for later snapshots).
\end{itemize}

\subsection*{All Games Page}

\begin{itemize}
  \item Shows a table or list of games with sport, home team, away team,
        date, and time.
  \item Contains a filter control (such as a dropdown) to limit games by sport.
  \item In later snapshots, includes a favorite icon/button to mark a game as
        a favorite.
\end{itemize}

\subsection*{My Favorites Page (planned)}

\begin{itemize}
  \item Shows only the games that the user has marked as favorites.
  \item Allows the user to remove a game from favorites.
\end{itemize}

\subsection*{Admin Page (planned)}

\begin{itemize}
  \item Allows project members to create, edit, or delete games.
  \item Includes a form with fields for sport, teams, date, and time.
\end{itemize}

\section{Current Status (Checkpoint 1)}

As of Snapshot 2 (Checkpoint 1), the focus is on implementing the
\textbf{All Games} view. A basic page for listing games and filtering by
sport is being developed using sample data. Favorites and admin features
will be added and documented in later snapshots.

\end{document}
