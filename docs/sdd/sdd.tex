\documentclass[12pt]{article}
\usepackage[margin=1in]{geometry}
\usepackage{hyperref}
\usepackage{longtable}

\title{Software Design Document (SDD)\\
SportHub: Multi-Sport Scores \& Favorites Tracker}
\author{Team 4: Miguel, Oscar, Florencio, Jesus}
\date{\today}

\begin{document}

\maketitle
\thispagestyle{empty}
\newpage

\tableofcontents
\newpage

\section*{Version History}
\addcontentsline{toc}{section}{Version History}
\begin{longtable}{|p{0.18\linewidth}|p{0.18\linewidth}|p{0.55\linewidth}|}
\hline
\textbf{Version} & \textbf{Date} & \textbf{Description} \\
\hline
0.1 & 2024-11-24 & Snapshot 1: Initial draft and base architecture (core pages and workflow). \\
\hline
0.2 & 2024-12-05 & Snapshot 2: Schedule view and filters using mock data; defined core REST endpoints and initial Docker layout. \\
\hline
0.3 & 2025-12-10 & Snapshot 3: Favorites flow design, API planning, updated UI notes. \\
\hline
1.0 & 2025-12-11 & Snapshot 4: Final version; polished favorites and filters, updated containers and documentation. \\
\hline
\end{longtable}
\newpage

\section{Introduction}

\subsection{Purpose}
SportHub is a web application that lets users see sports schedules and follow
their favorite teams. The goal of this document is to explain how the system is
designed: what the main parts are, how they connect, and how the website is
supposed to work. This SDD is used by our group while we build the project and
by the instructor when reviewing how we planned the system.

\subsection{Intended Audience}
This document is written for:
\begin{itemize}
  \item The CS3338 instructor and teaching assistants who grade the project.
  \item Our team members who are working on the SportHub code and tests.
  \item Anyone in the future who might want to understand or extend this project.
\end{itemize}

\subsection{Overview}
SportHub focuses on showing game schedules for different sports (for example
NBA, NFL, soccer and MLB). Fans can create an account, mark teams as favorites,
and see a personalized list of upcoming games for those teams. An admin user
can manage teams and games in the database.

The rest of this document describes:
\begin{itemize}
  \item The overall system architecture (frontend, backend, database, workflow).
  \item The user interface and how people move through the pages.
  \item The basic database design used by the application.
  \item A glossary of terms and references we used while designing the system.
\end{itemize}

\section{System Architecture}

\subsection{High-Level Design}
SportHub uses a common three-layer web design:

\begin{itemize}
  \item \textbf{Frontend (client)} – A web interface built as a single-page
        application (for example using React). It runs in the user's browser
        and shows the pages like Home, Game Details, Login, and My Favorites.
        The frontend talks to the backend by sending HTTP requests and
        receiving JSON data.
  \item \textbf{Backend API (server)} – A Node.js/Express application that
        contains the main logic. It exposes REST endpoints such as
        \texttt{/api/games}, \texttt{/api/teams}, \texttt{/api/auth/register},
        \texttt{/api/auth/login}, \texttt{/api/favorites},
        \texttt{/api/favorites/:teamId}, and \texttt{/api/favorites/games}.
        It checks requests, talks to the database, and sends back responses to
        the frontend.
  \item \textbf{Database} – A PostgreSQL database that stores persistent data:
        users, teams, games, and which teams each fan has favorited via a
        \texttt{favorites} junction table.
  \item \textbf{Runtime stack} – Node.js 20 with Express
        and CORS middleware for the API, plus nginx to serve the static
        frontend container. During development, favorites may persist locally
        until database wiring is completed.
\end{itemize}

All of these parts are run as Docker containers and are managed with
\texttt{docker-compose}. A typical setup includes a \texttt{frontend}
container, a \texttt{backend} container, a \texttt{db} container, and
optionally an \texttt{nginx} container used as a reverse proxy.

\subsection{Main Workflow}
A typical workflow for a fan user looks like this:

\begin{enumerate}
  \item The user opens the SportHub URL in a browser.  
        The browser downloads the frontend code.
  \item The frontend automatically calls the backend endpoint
        \texttt{/api/games} with a date and sport filter to get the list of
        games for that day.
  \item The backend queries the database for games that match the filters and
        returns them as JSON.
  \item The frontend displays these games on the Home page as a list of cards.
  \item If the user wants more features, they click \textbf{Register} and send
        their email and password to \texttt{/api/auth/register}. The backend
        creates a new user record in the database.
  \item When the user logs in, the backend verifies the password and returns a
        session or token. The frontend stores this so it knows the user is a
        Fan.
  \item As a logged-in Fan, the user can click a star icon next to a team to
        favorite it. The frontend sends a request to \texttt{/api/favorites}
        to save that team as a favorite for the current user.
  \item When the Fan visits the \textbf{My Favorites} page, the frontend calls
        \texttt{/api/favorites/games}. The backend finds all favorite teams for
        that user, finds games involving those teams, and returns the list.
  \item The frontend shows a personalized schedule based on those favorite
        teams.
\end{enumerate}

\subsection{Snapshot 2 Implementation Notes}
\begin{itemize}
  \item Implemented the initial All Games / schedule view using mock data served by the Express backend.
  \item Added basic sport and date filters on the Home page so users can narrow down the schedule.
  \item Defined the core REST endpoints for games (\texttt{/api/games}) and teams (\texttt{/api/teams}).
  \item Created the first Docker setup with separate frontend and backend services.
\end{itemize}

\subsection{Snapshot 3 Implementation Notes}
\begin{itemize}
  \item Favorites UI added to the All Games page (star/button toggle) with a
        basic My Favorites view that lists favorited games.
  \item Favorites are stored on the frontend for now (in-memory or local
        storage) while backend endpoints for persistence are designed.
  \item Planned endpoints include \texttt{/api/favorites} for toggling and
        \texttt{/api/favorites/games} for retrieving favorite games; both will
        extend the existing Express service.
  \item Frontend and backend remain containerized via \texttt{docker/Dockerfile.frontend}
        and \texttt{docker/Dockerfile.backend}.
  \item TestRail coverage expanded to include favorite/unfavorite and viewing
        the favorites list; results will be tracked in the Snapshot 3 report.
\end{itemize}

\subsection{Snapshot 4 Final Notes}
\begin{itemize}
  \item Polished navigation and styling so that all pages share the same layout and header.
  \item Verified the favorites flow (add/remove favorites, view My Favorites) and updated TestRail coverage for the final run.
  \item Finalized the Docker configuration so frontend and backend can be started together via \texttt{docker-compose}.
  \item Updated all project documents (SDD, SRS, user manual, design spec, snapshot objectives) to match the final system.
\end{itemize}

\subsection{Site Breakdown}
The main parts of the SportHub site are:

\begin{itemize}
  \item \textbf{Home / Schedule Page} – Shows a list of games. Users can choose
        a sport (All, NBA, NFL, Soccer, etc.) and a date filter. Each game has
        basic information and a button to view details.
  \item \textbf{Teams Page} – Lists teams by sport, shows basic info, and lets
        logged-in Fans add or remove a team from their favorites.
  \item \textbf{Game Details Page} – Shows more information about a single
        game, such as teams, league, game time, and location. It may also show
        a placeholder for scores.
  \item \textbf{Login / Register Pages} – Simple forms for creating an account
        and logging in as a Fan.
  \item \textbf{My Favorites Page} – Visible only to logged-in Fans. Shows the
        teams they have favorited and a list of upcoming games for those teams.
  \item \textbf{Admin Area} (simple) – A restricted area for an Admin user to
        add or edit teams and games.
\end{itemize}

These pages are connected by a shared navigation bar so users can move between
them easily.

\section{User Interface}

\subsection{Home Page}
When a user first opens SportHub, they see the Home page.

\begin{itemize}
  \item At the top there is a small filter area where the user can choose:
        \begin{itemize}
          \item The sport (for example All, NBA, NFL, Soccer).
          \item The date (for example Today or a specific calendar date).
        \end{itemize}
  \item Under the filters there is a list of games. Each game card shows:
        \begin{itemize}
          \item Home team vs. away team.
          \item League name.
          \item Game start time.
          \item A button or link labeled ``View Details''.
        \end{itemize}
  \item If the user is logged in as a Fan, the Home page may also show a star
        icon next to each team to mark it as a favorite.
\end{itemize}

\subsection{Game Details Page}
When the user clicks on a game from the Home page:

\begin{itemize}
  \item The Game Details page is opened.
  \item It shows the two teams, the league, the game date and time, and the
        location of the game.
  \item If scores are available (even if they are just mock values for the
        project), they can be shown here as home score and away score.
  \item A back link or button lets the user return to the previous page.
\end{itemize}

\subsection{Teams Page}
\begin{itemize}
  \item Shows teams grouped by sport with name and optional logo.
  \item Logged-in Fans can click to add or remove a team from their favorites
        (using \texttt{/api/favorites} requests).
  \item Provides a quick way to build the My Favorites list.
\end{itemize}

\subsection{Login and Registration Pages}
\begin{itemize}
  \item The \textbf{Login} page has fields for email and password and a button
        to submit the form. If the credentials are correct, the user becomes
        a logged-in Fan (calling \texttt{/api/auth/login}).
  \item The \textbf{Register} page lets a new user create an account by
        entering an email and password (calling \texttt{/api/auth/register}).
        After registration they can log in and start using favorites.
  \item Error messages are shown for invalid input (for example, missing email
        or wrong password).
\end{itemize}

\subsection{My Favorites Page}
\begin{itemize}
  \item This page is only available when the user is logged in as a Fan.
  \item It shows a list of the Fan's favorite teams.
  \item Below that, it shows upcoming games that involve any of those teams.
  \item The user can unfavorite a team from this page if they no longer want
        to follow it.
\end{itemize}

\subsection{Database Overview}
The user interface is backed by a simple relational database. The main tables
are:

\begin{itemize}
  \item \textbf{Users} – Stores accounts for Fans and Admins.
        Typical columns: \texttt{id}, \texttt{email}, \texttt{password\_hash},
        \texttt{role}, \texttt{created\_at}.
  \item \textbf{Teams} – Stores information about teams.
        Columns: \texttt{id}, \texttt{name}, \texttt{league}, \texttt{city},
        \texttt{is\_active}.
  \item \textbf{Games} – Stores each scheduled game.
        Columns: \texttt{id}, \texttt{home\_team\_id}, \texttt{away\_team\_id},
        \texttt{league}, \texttt{game\_date}, \texttt{location},
        \texttt{home\_score}, \texttt{away\_score}, \texttt{is\_final}.
  \item \textbf{Favorites} – Links Fans to the teams they follow.
        Columns: \texttt{id}, \texttt{user\_id}, \texttt{team\_id}.
\end{itemize}

The backend uses these tables to build the lists shown on the pages. For
example, the Home page reads rows from the \texttt{Games} and \texttt{Teams}
tables, and the My Favorites page joins \texttt{Favorites}, \texttt{Users},
and \texttt{Teams}.

\section{Glossary}

\begin{longtable}{|p{0.25\linewidth}|p{0.65\linewidth}|}
\hline
\textbf{Term / Acronym} & \textbf{Definition} \\
\hline
API & Application Programming Interface. The set of HTTP endpoints the frontend
uses to talk to the backend. \\
\hline
DB & Database. In this project it refers to the PostgreSQL instance. \\
\hline
Fan & A registered user who can log in, mark favorite teams, and see a personalized schedule. \\
\hline
Guest & A user who is not logged in and can only browse public pages. \\
\hline
Admin & A user role with permission to manage teams and games in the system. \\
\hline
REST & Representational State Transfer, a common style for web APIs. \\
\hline
SPA & Single-Page Application, a web app that loads once and updates content
dynamically without full page reloads. \\
\hline
JSON & JavaScript Object Notation, a text format used to send data between
frontend and backend. \\
\hline
\end{longtable}

\section{References}

\begin{itemize}
  \item React documentation: \url{https://react.dev/}
  \item Node.js documentation: \url{https://nodejs.org/en/docs}
  \item Express.js documentation: \url{https://expressjs.com/}
  \item PostgreSQL documentation: \url{https://www.postgresql.org/docs/}
  \item Docker documentation: \url{https://docs.docker.com/}
\end{itemize}

\end{document}
